% =====================================================
% Curso de LaTeX (ENCiT)
% =====================================================
\documentclass{article}
\usepackage[utf8]{inputenc}
\usepackage[T1]{fontenc}
%\usepackage[spanish]{babel}
%\renewcommand{\tablename}{Tabla}
\usepackage{booktabs} % Para líneas horizontales de mejor calidad

\begin{document}

\section{Curso de \LaTeX: Ejemplo de Tablas Flotantes}

En este documento verás distintos ejemplos de cómo insertar y 
referenciar tablas en \LaTeX. Observa cómo la posición de las leyendas 
(\emph{caption}), el uso de distintos \emph{layouts} y los parámetros 
de flotación (\texttt{[h]}, \texttt{[t]}, \texttt{[b]}, \texttt{[p]}, \texttt{!}, etc.) 
influyen en el resultado final.

\subsection{Tabla con leyenda debajo (posición preferente \texttt{[ht]})}

En la Tabla~\ref{tab:simple} se muestra un ejemplo sencillo en el que la 
leyenda se coloca \textbf{debajo} de la tabla.

\begin{table}[ht]
    \centering
    \begin{tabular}{c|c|c}
    \hline
    \textbf{Columna 1} & \textbf{Columna 2} & \textbf{Columna 3} \\
    \hline
    A & B & C \\
    D & E & F \\
    \hline
    \end{tabular}
    \caption{Tabla sencilla con caption debajo}
    \label{tab:simple}
\end{table}

Como ves, \LaTeX\ decide colocar la tabla cerca de este bloque de texto,
pero siempre intentando no romper la composición de la página.

\subsection{Tabla con leyenda arriba (forzando \texttt{[h!]})}

La Tabla~\ref{tab:arriba} ilustra cómo poner la leyenda \textbf{encima} 
de la tabla y además usar \texttt{[h!]} para forzar al máximo que aparezca 
justo aquí:

\begin{table}[h!]
    \caption{Leyenda colocada arriba de la tabla}
    \label{tab:arriba}
    \centering
    \begin{tabular}{l r}
        \toprule
        \textbf{Producto} & \textbf{Cantidad} \\
        \midrule
        Manzanas & 10 \\
        Naranjas & 5 \\
        Peras & 12 \\
        \bottomrule
    \end{tabular}
\end{table}

Aunque uses \texttt{[h!]}, \LaTeX\ mantiene libertad de mover la tabla
en caso de que no se ajuste bien a la página.

\subsection{Tabla con layout más elaborado (\texttt{booktabs} y \texttt{multicolumn})}

En la Tabla~\ref{tab:multicol} se muestra un ejemplo con múltiples columnas,
divisiones claras y el uso de \textbackslash multicolumn para centrar un título
sobre varias columnas.

\begin{table}[ht]
    \centering
    \caption{Tabla con booktabs y \texttt{multicolumn}}
    \label{tab:multicol}
    \begin{tabular}{lcc}
    \toprule
    \multicolumn{1}{c}{\textbf{Item}} & \textbf{Cantidad} & \textbf{Precio} \\
    \midrule
    Manzanas & 10 & \$15 \\
    Naranjas & 5  & \$8  \\
    \midrule
    \textbf{Total} & 15 & \$23 \\
    \bottomrule
    \end{tabular}
\end{table}

La opción \texttt{[ht]} indica que preferimos que LaTeX la ubique aquí (here)
o en la parte superior (top) de la página.

\subsection{Tabla en página separada (\texttt{[p]})}

A veces, cuando la tabla es grande o queremos colocarla en 
una hoja aparte, podemos usar \texttt{[p]}. Observa la 
Tabla~\ref{tab:page}.

\begin{table}[p]
    \centering
    \caption{Tabla colocada en una página de flotantes (\texttt{[p]})}
    \label{tab:page}
    \begin{tabular}{c c}
    \toprule
    \textbf{X} & \textbf{Y} \\
    \midrule
    1 & 2 \\
    3 & 4 \\
    5 & 6 \\
    \bottomrule
    \end{tabular}
\end{table}

Si el documento no es muy extenso, es posible que \LaTeX\ la sitúe 
al final de tu archivo; de cualquier forma, quedará 
\textbf{separada en una página para flotantes}.

\subsection{Tips de Tablas Referenciadas}

La versión más comúnmente aceptada de estilo para una tesis sugiere ubicar la tabla 
en la parte superior (\emph{top}) de la página y colocar la leyenda (\emph{caption}) 
arriba. De esta forma, el lector identifica inmediatamente el contenido de la tabla 
sin necesidad de desplazar la mirada hacia abajo. A continuación, algunos consejos útiles:

\begin{itemize}
  \item Usa el posicionador \textbf{[t]}: Para indicar que tu tabla se ubique preferentemente 
  en la parte superior de la página. Puedes reforzar esta preferencia con \texttt{[t!]} 
  si deseas forzar aún más su ubicación.
  
  \item Coloca la \textbackslash caption antes del entorno \texttt{tabular}: Así 
  garantizarás que la leyenda aparezca arriba de la tabla. Por ejemplo:
  \begin{verbatim}
  \begin{table}[t]
      \caption{Tu leyenda aquí}
      \label{tab:ejemplo}
      \centering
      \begin{tabular}{...}
          ...
      \end{tabular}
  \end{table}
  \end{verbatim}

  \item Etiqueta y referencia (\textbackslash label y \textbackslash ref): 
  Incluye el \texttt{\textbackslash label\{\}} dentro del entorno \texttt{table}, 
  de preferencia justo después de \texttt{\textbackslash caption}. Luego, en tu texto, 
  usa \texttt{Tabla~\textbackslash ref\{tab:ejemplo\}} para hacer referencia. 
  Esto mantiene coherencia y facilita la actualización automática del número de tabla.

  \item Cuida la consistencia en todo el documento: Si decides colocar todas 
  las leyendas arriba de las tablas, sé consistente a lo largo de tu tesis. 
  La homogeneidad contribuye a una mejor presentación.

  \item Revisa los lineamientos de tu Comité Académico, en caso de no haberlos, decide la opción que te parezca más adecuada para tu trabajo, junto con tu asesor/a/e de tesis.


  \item Evita saturar la tabla: Incluye únicamente los datos esenciales. Para 
  información muy amplia, considera otros enfoques (p.ej., \texttt{longtable} para 
  tablas que se extienden a varias páginas).

  \item Usa paquetes útiles:
    \begin{itemize}
      \item \texttt{booktabs} para líneas horizontales de calidad (\texttt{\textbackslash toprule}, 
      \texttt{\textbackslash midrule}, \texttt{\textbackslash bottomrule}).
      \item \texttt{babel} (o \texttt{polyglossia}) para usar “Tabla” en lugar de “Table”.
      \item \texttt{tabularx} o \texttt{tabu} para ajustar el ancho de columnas 
      automáticamente.
    \end{itemize}

\end{itemize}

En resumen, colocar la tabla en la parte superior de la página con la leyenda arriba 
es una práctica estándar en muchos manuales de estilo para tesis, pero siempre revisa 
las directrices específicas de tu institución o disciplina.


\end{document}
