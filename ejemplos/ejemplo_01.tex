\documentclass[12pt,a4paper]{article} % Define el tipo de documento y tamaño de fuente

\usepackage[utf8]{inputenc} % Codificación de caracteres
\usepackage[spanish]{babel} % Idioma del documento
\usepackage{graphicx}       % Inserción de gráficos


% Comienza el documento
\begin{document}

\section{Introducción} % Título de una sección
Este es mi primer documento elaborado en \LaTeX. Con \LaTeX, puedes destacar texto usando \textbf{negritas}, \textit{cursiva} o \underline{subrayado}. 

También puedes organizar información usando listas ordenadas, no ordenadas y anidadas.

\section{Ejemplos de Listas} % Otra sección

\subsection{Lista No Ordenada} % Subsección
Las listas no ordenadas se crean con el entorno `itemize`:
\begin{itemize}
    \item Este es el primer elemento.
    \item Este es el segundo elemento.
    \item Puedes incluir tantos elementos como desees.
\end{itemize}

\subsection{Lista Ordenada} % Subsección
Las listas ordenadas utilizan el entorno `enumerate`:
\begin{enumerate}
    \item Este es el primer elemento de la lista.
    \item Este es el segundo elemento.
    \item El formato numérico es automático.
\end{enumerate}

\subsection{Listas Anidadas} % Subsección
Puedes combinar listas ordenadas y no ordenadas para crear listas anidadas:
\begin{enumerate}
    \item Primer elemento de la lista principal.
    \begin{itemize}
        \item Sub-elemento no ordenado.
        \item Otro sub-elemento.
    \end{itemize}
    \item Segundo elemento de la lista principal.
    \begin{enumerate}
        \item Sub-elemento ordenado.
        \item Otro sub-elemento ordenado.
    \end{enumerate}
\end{enumerate}

\section{Conclusión} % Última sección
Aprender \LaTeX es una herramienta \textbf{valiosa} para crear documentos largos como tesis, artículos científicos o reportes técnicos. Además, permite incluir \textit{destacados estilísticos} y estructuras complejas como listas anidadas sin perder consistencia visual.  

\end{document} % Termina el documento

