\documentclass[12pt]{article}

% Paquetes necesarios
\usepackage{amsmath} % Para ecuaciones avanzadas
\usepackage{amssymb} % Para símbolos matemáticos

% Metadatos
\title{Ejercicio sobre Uso de Ecuaciones en \LaTeX}
\author{Tu Nombre}
\date{\today}

\begin{document}

\maketitle



\section{La ecuación de Planck}


Max Planck calculó la densidad espectral, también llamada irradiancia o flujo espectral de un cuerpo negro con temperatura T. Esta función, que se denomina función de Planck ($B$), se puede escribir como:

\begin{equation}
B(\nu,T)=\frac{h\nu^3}{c^2}\frac{1}{e^{\frac{h\nu}{kT}}-1}
\label{eq:planck}
\end{equation}

\noindent donde $h$ es la constante de Planck (6.626$\times$10$^{-34}$ J s), $k$
 es la constante de Boltzmann 1.380649$\times$10$^{-23}$ J K$^{-1}$ y $T$ es la temperatura en K. 
  
La ley o ecuación de Planck \eqref{eq:planck} nos dice que la densidad de radiación de un cuerpo negro en equilibrio depende de su temperatura y de su frecuencia.

\end{document}