\documentclass[12pt,a4paper]{article} % Define el tipo de documento y tamaño de fuente

\usepackage[utf8]{inputenc} % Codificación de caracteres
\usepackage[spanish]{babel} % Idioma del documento
\usepackage{graphicx}       % Inserción de gráficos


% Metadatos del documento
\title{El primer ejercicio en \LaTeX}
\author{Mi Nombre (o sea el nombre de quién está haciendo este ejercicio)}
\date{23 de febrero de 2025} % Inserta automáticamente la fecha actual

% Comienza el documento
\begin{document}

\maketitle % Genera el título usando los metadatos anteriores

\section{Introducción} % Título de una sección
Este es un primer ejercicio en \LaTeX. 
La tesis es quizás uno de los trabajos \textbf{más difíciles} de realizar en una licenciatura. La \underline{primer} dificultad aparece debido a la longitud del documento, ya que durante la carrera uno se acostumbra a documentos de 5 a 10 páginas. Sin embargo, una tesis puede llegar a tener hasta \textit{100 páginas}, o más. La \underline{segunda} dificultad aparece debido a lo abrumador que puede llegar a ser entregar un trabajo que toma meses de preparación. Es difícil percibir los avances y se puede desgastar uno con facilidad, el que ahora es conocido como \textit{burn-out}. 



\section{Tips} % Otra sección

A continuación algunos tips para solventar esta dificultad.

Las listas no ordenadas se crean con el entorno `itemize`:
\begin{itemize}
    \item Organización y rutina.
    \item Listas de pendientes.
    \item Retroalimentación de tutores o pares.
\end{itemize}


\end{document} % Termina el documento

